\begingroup\fontsize{8}{10}\selectfont

\begin{ThreePartTable}
\begin{TableNotes}
\item[a] Protein localisation was inferred from sequence 
                         similarity with \textit{S. cerevisiae} homolog as 
                         annotated in the Candida Genome Database (48, 49).
\item[b] Protein localisation was obtained from the GO Cellular
                        Component annotation in the \textit{C. albicans} 
                        UniProt reference proteome UP000000559 (58).
\item[c] Presence of transmembrane domains and absence of a 
                        signal peptide was predicted using TOPCONS2 (59).
\item[d] The name Evp1 for the protein encoded by orf19.6741 
                        was proposed in the present study.
\end{TableNotes}
\begin{longtable}[t]{lllllllll}
\caption{\label{tab:}\textbf{Candidate positive protein markers for 
             \textit{C. albicans} EVs.} This list of proteins consists of 
             those that were found to be exclusive to EVs or significantly 
             enriched in EVs across the four \textit{C. albicans} strains 
             examined in this study. Proteins are grouped according to their 
             subcellular localisation as annotated in the Candida Genome 
             Database (candidagenome.org) (ref:footnote1) unless otherwise 
             indicated. The log\textsubscript{2} ratio of the abundance 
             (mean MaxQuant LFQ intensity) of each protein in EVs compared to 
             whole cell lysate (WCL) for each strain is listed. ``ex" 
             indicates where a protein was only quantified in the EV fraction 
             and not the WCL for that strain. The ``TM" column indicates the 
             number of transmembrane domains for each protein as annotated in 
             UniProt. ``SP" indicates whether a protein is annotated as 
             having a signal peptide according to UniProt. ``VDM" shows 
             whether a protein has previously been detected in vesicle-depleted 
             culture media (i.e. the proteins may also be in the soluble secretome) 
             (31). Underlined proteins are those identified as the 
             best candidates for positive EV markers according to the 
             criteria depicted in Supplementary Figure S1.}\\
\toprule
\multicolumn{2}{c}{ } & \multicolumn{4}{c}{log\textsubscript{2}(fold change) EV vs WCL} & \multicolumn{2}{c}{ } \\
\cmidrule(l{3pt}r{3pt}){3-6}
Name & Function & DAY Y & A9 & A1 & DAY B & TM & SP & VDM\\
\midrule
\addlinespace[0.3em]
\multicolumn{9}{l}{\textbf{Plasma membrane}}\\
\hspace{1em}\underline{ARF3} & \underline{Arf family GTPase<sup>a</sup>} & \underline{ex} & \underline{1.69} & \underline{ex} & \underline{3.45} & \underline{} & \underline{} & \underline{}\\
\hspace{1em}\underline{CDC42} & \underline{Rho family GTPase} & \underline{ex} & \underline{ex} & \underline{ex} & \underline{ex} & \underline{} & \underline{} & \underline{}\\
\hspace{1em}CDR1;CDR2 & Multidrug transporter of ABC superfamily & ex & ex & ex & 11.19 & 12;12 &  & \\
\hspace{1em}CHS3 & Major chitin synthase of yeast and hyphae & ex & 4.12 & ex & ex & 5 &  & \\
\hspace{1em}ENA21 & Predicted P-type ATPase sodium pump<sup>a</sup> & ex & ex & ex & ex & 9 &  & \\
\hspace{1em}\underline{FAA4} & \underline{Long-chain fatty acid-CoA ligase<sup>a</sup>} & \underline{2.89} & \underline{2.24} & \underline{2.28} & \underline{1.45} & \underline{} & \underline{} & \underline{}\\
\hspace{1em}\underline{FET34} & \underline{Multicopper feroxidase} & \underline{ex} & \underline{ex} & \underline{ex} & \underline{6.18} & \underline{1} & \underline{Y} & \underline{}\\
\hspace{1em}GAP4 & High-affinity S-adenosylmethionine permease & ex & ex & ex & 4.28 & 12 &  & \\
\hspace{1em}GSC1 & 1,3-beta-glucan synthase & 2.57 & 3.45 & 5.97 & 8.01 & 15 &  & \\
\hspace{1em}HGT1 & High-affinity MFS glucose transporter & ex & ex & ex & ex & 12 &  & \\
\hspace{1em}HGT6 & Putative high-affinity MFS glucose transporter & 4.04 & 4.78 & 6.53 & 5.68 & 11 &  & \\
\hspace{1em}HGT7 & Putative MFS glucose transporter & ex & 5.95 & ex & ex & 11 &  & \\
\hspace{1em}\underline{MTS1} & \underline{Sphingolipid C9-methyltransferase} & \underline{2.20} & \underline{1.86} & \underline{1.41} & \underline{2.13} & \underline{2} & \underline{} & \underline{}\\
\hspace{1em}\underline{NCE102} & \underline{Non-classical protein export protein} & \underline{3.34} & \underline{2.20} & \underline{ex} & \underline{5.32} & \underline{4} & \underline{} & \underline{}\\
\hspace{1em}\underline{EVP1<sup>d</sup>} & \underline{\textit{S. cerevisiae} ortholog is Pun1, plasma membrane protein<sup>a</sup>} & \underline{ex} & \underline{ex} & \underline{ex} & \underline{ex} & \underline{3} & \underline{} & \underline{Y}\\
\hspace{1em}PHM7 & Putative ion transporter & ex & ex & ex & ex & 11 &  & \\
\hspace{1em}PMA1 & Plasma membrane ATPase & 5.65 & 4.38 & 5.60 & 6.46 & 8 &  & \\
\hspace{1em}\underline{RAC1} & \underline{G-protein of RAC subfamily} & \underline{ex} & \underline{ex} & \underline{ex} & \underline{3.92} & \underline{} & \underline{} & \underline{}\\
\hspace{1em}\underline{RHO1} & \underline{Rho family GTPase} & \underline{2.81} & \underline{3.68} & \underline{3.74} & \underline{4.03} & \underline{} & \underline{} & \underline{}\\
\hspace{1em}\underline{RHO3} & \underline{Rho family GTPase} & \underline{ex} & \underline{ex} & \underline{ex} & \underline{ex} & \underline{} & \underline{} & \underline{}\\
\hspace{1em}\underline{SSO2} & \underline{Plasma membrane t-SNARE<sup>a</sup>} & \underline{ex} & \underline{1.88} & \underline{3.57} & \underline{ex} & \underline{1} & \underline{} & \underline{}\\
\hspace{1em}\underline{SUR7} & \underline{Protein required for normal cell wall, plasma membrane<sup>c</sup>} & \underline{ex} & \underline{ex} & \underline{ex} & \underline{7.04} & \underline{4} & \underline{} & \underline{}\\
\hspace{1em}\underline{YCK2} & \underline{\textit{S. cerevisiae} ortholog is Yck2, casein kinase} & \underline{ex} & \underline{4.42} & \underline{ex} & \underline{ex} & \underline{} & \underline{} & \underline{}\\
\addlinespace[0.3em]
\multicolumn{9}{l}{\textbf{Cell wall, cell surface}}\\
\hspace{1em}BGL2 & 1,3-beta-glucanosyltransferase & 7.26 & 9.44 & 11.08 & ex &  & Y & Y\\
\hspace{1em}CRH11 & GPI-anchored cell wall transglycosylase & ex & ex & ex & ex &  & Y & Y\\
\hspace{1em}ECM33 & GPI-anchored cell wall protein & 4.39 & 5.28 & 8.31 & 7.03 &  & Y & Y\\
\hspace{1em}\underline{GPD2} & \underline{Glycerol-3-phosphate dehydrogenase} & \underline{ex} & \underline{3.48} & \underline{2.39} & \underline{1.30} & \underline{} & \underline{} & \underline{}\\
\hspace{1em}MP65 & Cell surface mannoprotein & ex & ex & ex & ex &  & Y & Y\\
\hspace{1em}MSB2 & Mucin family adhesin-like protein & ex & ex & ex & ex & 1 & Y & Y\\
\hspace{1em}PGA4 & 1,3-beta-glucanosyltransferase & ex & ex & ex & ex &  & Y & Y\\
\hspace{1em}PGA52 & GPI-anchored cell surface protein of unknown function & ex & ex & ex & ex &  & Y & Y\\
\hspace{1em}\underline{PHR1} & \underline{Cell surface glycosidase} & \underline{ex} & \underline{ex} & \underline{ex} & \underline{5.65} & \underline{} & \underline{Y} & \underline{}\\
\hspace{1em}PHR2 & Glycosidase & ex & 3.51 & 6.44 & ex &  & Y & Y\\
\hspace{1em}PLB4.5 & Phospholipase B & ex & ex & ex & ex &  & Y & Y\\
\hspace{1em}SAP9 & Secreted aspartyl protease & ex & ex & ex & ex & 1 & Y & Y\\
\hspace{1em}YWP1 & Secreted yeast cell wall protein & 2.37 & 3.51 & 6.17 & ex &  & Y & Y\\
\addlinespace[0.3em]
\multicolumn{9}{l}{\textbf{Endoplasmic reticulum}}\\
\hspace{1em}ERO1 & ER oxidoreductin & ex & ex & 2.32 & 1.68 &  & Y & Y\\
\hspace{1em}\underline{orf19.1054} & \underline{\textit{S. cerevisiae} ortholog is Pom33, transmembrane nucleoporin<sup>b</sup>} & \underline{5.66} & \underline{ex} & \underline{ex} & \underline{2.51} & \underline{4} & \underline{} & \underline{}\\
\hspace{1em}\underline{orf19.2168.3} & \underline{\textit{S. cerevisiae} ortholog is Yop1, reticulon-interacting protein<sup>a</sup>} & \underline{4.77} & \underline{1.98} & \underline{2.29} & \underline{1.81} & \underline{4} & \underline{} & \underline{}\\
\hspace{1em}\underline{orf19.3799} & \underline{\textit{S. cerevisiae} ortholog is Rtn1, reticulon protein<sup>b</sup>} & \underline{2.24} & \underline{1.40} & \underline{3.00} & \underline{1.29} & \underline{2} & \underline{} & \underline{}\\
\hspace{1em}SEC61 & ER protein-translocation complex subunit & 2.53 & 2.16 & 2.10 & 1.46 & 8 &  & \\
\addlinespace[0.3em]
\multicolumn{9}{l}{\textbf{Endosome, Golgi, transport vesicle}}\\
\hspace{1em}\underline{SEC4} & \underline{Rab family GTPase<sup>a</sup>} & \underline{1.88} & \underline{1.20} & \underline{1.86} & \underline{2.90} & \underline{} & \underline{} & \underline{}\\
\hspace{1em}\underline{YKT6} & \underline{Palmitoyltransferase, putative vacuolar SNARE complex protein<sup>a</sup>} & \underline{ex} & \underline{4.01} & \underline{5.46} & \underline{3.20} & \underline{} & \underline{} & \underline{}\\
\hspace{1em}\underline{YPT31} & \underline{Rab family GTPase<sup>a</sup>} & \underline{ex} & \underline{1.57} & \underline{1.33} & \underline{3.33} & \underline{} & \underline{} & \underline{}\\
\addlinespace[0.3em]
\multicolumn{9}{l}{\textbf{Vacuole}}\\
\hspace{1em}\underline{VAC8} & \underline{Protein involved in vacuolar inheritance<sup>a</sup>} & \underline{2.79} & \underline{ex} & \underline{ex} & \underline{3.83} & \underline{} & \underline{} & \underline{}\\
\addlinespace[0.3em]
\multicolumn{9}{l}{\textbf{Mitochondrion}}\\
\hspace{1em}MIR1 & Putative mitochondrial phosphate transporter<sup>a</sup> & 3.68 & 4.38 & 4.24 & 1.60 &  &  & \\
\hspace{1em}POR1 & Mitochondrial outer membrane porin<sup>a</sup> & 3.44 & 2.51 & 3.69 & 1.38 &  &  & \\
\bottomrule
\insertTableNotes
\end{longtable}
\end{ThreePartTable}
\endgroup{}
