\begingroup\fontsize{8}{10}\selectfont

\begin{ThreePartTable}
\begin{TableNotes}
\item[a] Protein localisation was inferred from sequence 
                         similarity with \textit{S. cerevisiae} homolog as 
                         annotated in the Candida Genome Database 
                         (ref:footnote1)
\item[b] Protein localisation was obtained from the GO Cellular
                        Component annotation in the \textit{C. albicans} 
                        UniProt reference proteome UP000000559 (ref:footnote2).
\item[c] Protein and has no Cellular Component annotation in 
                         the Candida Genome Database or UniProt reference 
                        proteome.
\end{TableNotes}
\begin{longtable}[t]{lllllllll}
\caption{\label{tab:}\textbf{Candidate negative protein markers for 
              \textit{C. albicans} EVs.} This list of proteins consists of 
              those that were found to be exclusive to whole cell lysate (WCL) 
              or significantly enriched in WCL across the four 
              \textit{C. albicans} strains examined in this study. Proteins 
              are grouped according to their 
              subcellular localisation as annotated in the Candida Genome 
              Database (candidagenome.org) (ref:footnote1) unless otherwise 
              indicated. The log\textsubscript{2} ratio of the abundance 
              (mean MaxQuant LFQ intensity) of each protein in EVs compared to 
              WCL for each strain is listed. A negative value indicates that a 
              protein was enriched in WCL compared to EV. ``ex" indicates 
              where a protein was only quantified in the WCL fraction and not 
              in EVs for that strain. The ``TM" column indicates the number 
              of transmembrane domains for each protein as annotated in 
              UniProtKB. ``SP" indicates whether a protein is annotated as 
              having a signal peptide according to UniProt. ``VDM" shows 
              whether a protein has previously been detected in 
              vesicle-depleted culture media (i.e. the proteins may also be in 
              the soluble secretome) (ref:table1). Underlined proteins are 
              those identified as the best candidates for negative
              EV markers according to the criteria depicted in 
              Supplementary Figure S1.}\\
\toprule
\multicolumn{2}{c}{ } & \multicolumn{4}{c}{log\textsubscript{2}(fold change) EV vs WCL} & \multicolumn{3}{c}{ } \\
\cmidrule(l{3pt}r{3pt}){3-6}
Name & Function & DAY Y & A9 & A1 & DAY B & TM & SP & VDM\\
\midrule
\addlinespace[0.3em]
\multicolumn{9}{l}{\textbf{Cytoplasm}}\\
\hspace{1em}ARO3 & Phospho-2-dehydro-3-deoxyheptonate aldolase<sup>b</sup> & -1.28 & -2.55 & -1.51 & -1.46 &  &  & \\
\hspace{1em}ARO8 & Aromatic transaminase of the Ehrlich fusel oil pathway<sup>a</sup> & -2.37 & -1.81 & ex & -2.36 &  &  & \\
\hspace{1em}GDB1 & Putative glycogen debranching enzyme & ex & ex & ex & -1.99 &  &  & \\
\hspace{1em}HOM2 & Aspartate-semialdehyde dehydrogenase<sup>b</sup> & ex & -3.41 & ex & ex &  &  & \\
\hspace{1em}orf19.1889 & Putative phosphoglycerate mutase family protein<sup>b</sup> & ex & -1.60 & ex & ex &  &  & \\
\hspace{1em}orf19.5943.1 & \textit{S. cerevisiae} ortholog is Stm1, regulates translation<sup>a</sup> & ex & -3.98 & -4.57 & ex &  &  & \\
\hspace{1em}orf19.6596 & Putative esterase<sup>a</sup> & ex & ex & ex & -2.90 &  &  & \\
\hspace{1em}orf19.7263 & Putative X-Pro aminopeptidase<sup>a</sup> & ex & ex & ex & ex &  &  & \\
\hspace{1em}SBP1 & \textit{S. cerevisiae} ortholog is Sbp1, eIF4G binding protein<sup>a</sup> & ex & -4.30 & -3.80 & -2.15 &  &  & \\
\hspace{1em}STI1 & HSP90 co-chaperone<sup>a</sup> & ex & -2.20 & ex & ex &  &  & \\
\hspace{1em}URA4 & Predicted succinate semialdehyde dehydrogenase<sup>a</sup> & ex & ex & ex & ex &  &  & \\
\hspace{1em}XKS1 & Putative xylulokinase<sup>b</sup> & ex & ex & ex & ex &  &  & \\
\hspace{1em}YNK1 & Nucleoside disphosphate kinase & -3.07 & -2.57 & -2.17 & -1.40 &  &  & \\
\addlinespace[0.3em]
\multicolumn{9}{l}{\textbf{Cytosol and mitochondria}}\\
\hspace{1em}ACH1 & Acetyl-CoA hydrolase<sup>b</sup> & -3.24 & -5.09 & -2.79 & -4.03 &  &  & \\
\hspace{1em}GLR1 & Glutathione reductase<sup>b</sup> & ex & ex & ex & -3.28 &  &  & \\
\hspace{1em}HSP60 & Heat shock protein 60<sup>b</sup> & -4.66 & -1.59 & -4.22 & -4.06 &  &  & \\
\addlinespace[0.3em]
\multicolumn{9}{l}{\textbf{Mitochondria}}\\
\hspace{1em}AAT1 & Aspartate aminotransferase<sup>a</sup> & ex & -2.11 & -1.48 & -5.70 &  &  & \\
\hspace{1em}BAT22 & Putative branched chain amino acid aminotransferase<sup>b</sup> & ex & -3.57 & -1.44 & -3.12 &  &  & \\
\hspace{1em}ETR1 & Putative 2-enoyl thioester reductase<sup>a</sup> & ex & ex & ex & ex &  &  & \\
\hspace{1em}GCV2 & Glycine decarboxylase P subunit<sup>a</sup> & ex & ex & ex & ex &  &  & \\
\hspace{1em}IDH1 & Isocitrate dehydrogenase subunit<sup>a</sup> & -2.67 & -2.48 & -1.85 & -2.22 &  &  & \\
\hspace{1em}IDH2 & Isocitrate dehydrogenase<sup>a</sup> & -3.21 & -2.88 & -3.54 & -3.37 &  &  & \\
\hspace{1em}IDP1 & Putative isocitrate dehydrogenase<sup>a</sup> & ex & ex & ex & ex &  &  & \\
\hspace{1em}ILV5 & Ketol-acid reductoisomerase<sup>b</sup> & -3.68 & -2.62 & -2.27 & -5.48 &  &  & \\
\hspace{1em}KGD2 & Putative dihydrolipoamide S-succinyltransferase<sup>a</sup> & ex & -2.20 & -2.04 & ex &  &  & \\
\hspace{1em}\underline{LPD1} & \underline{Dihydrolipoamide dehydrogenase} & \underline{ex} & \underline{-7.49} & \underline{-4.08} & \underline{-7.91} & \underline{} & \underline{} & \underline{}\\
\hspace{1em}NIF3 & \textit{S. cerevisiae} ortholog is Nif3, mitochondrial protein<sup>a</sup> & ex & ex & ex & ex &  &  & \\
\hspace{1em}orf19.2966 & Predicted dienelactone hydrolase domain<sup>a</sup> & ex & ex & ex & ex &  &  & \\
\hspace{1em}orf19.449 & Putative phosphatidyl synthase<sup>b</sup> & ex & ex & ex & ex &  &  & \\
\hspace{1em}orf19.7215.3 & \textit{S. cerevisiae} ortholog is Hsp10, mitochondrial co-chaperonin<sup>a</sup> & ex & ex & ex & ex &  &  & \\
\hspace{1em}PDX3 & Pyridoxamine-phosphate oxidase<sup>a</sup> & ex & -1.85 & ex & ex &  &  & \\
\hspace{1em}\underline{SOD2} & \underline{Superoxide dismutase} & \underline{ex} & \underline{ex} & \underline{ex} & \underline{ex} & \underline{} & \underline{} & \underline{}\\
\addlinespace[0.3em]
\multicolumn{9}{l}{\textbf{Nucleus}}\\
\hspace{1em}DOT5 & Thioredoxin peroxidase<sup>a</sup> & ex & ex & ex & ex &  &  & \\
\hspace{1em}NHP6A & Non-histone chromosomal protein 6<sup>a</sup> & ex & ex & ex & ex &  &  & \\
\addlinespace[0.3em]
\multicolumn{9}{l}{\textbf{Vacuole}}\\
\hspace{1em}AMS1 & Putative alpha-mannosidase<sup>a</sup> & ex & ex & ex & -2.23 &  &  & \\
\hspace{1em}\underline{APR1} & \underline{Vacuolar aspartic proteinase} & \underline{-1.83} & \underline{ex} & \underline{ex} & \underline{-1.07} & \underline{} & \underline{Y} & \underline{}\\
\hspace{1em}\underline{CPY1} & \underline{Carboxypeptidase Y} & \underline{ex} & \underline{ex} & \underline{ex} & \underline{-1.59} & \underline{} & \underline{Y} & \underline{}\\
\hspace{1em}\underline{LAP41} & \underline{Putative aminopeptidase yscI precursor} & \underline{ex} & \underline{-3.96} & \underline{-1.80} & \underline{-2.64} & \underline{} & \underline{} & \underline{}\\
\addlinespace[0.3em]
\multicolumn{9}{l}{\textbf{Cell wall, cell surface, fungal biofilm matrix}}\\
\hspace{1em}CPR3 & Putative peptidyl-prolyl cis-trans isomerase & ex & -2.59 & ex & ex &  &  & \\
\hspace{1em}GCY1 & Glycerol 2-dehydrogenase & ex & ex & ex & ex &  &  & \\
\hspace{1em}GLX3 & Glutathione-independent glyoxalase & ex & -6.21 & ex & -1.49 &  &  & \\
\hspace{1em}\underline{GPM1} & \underline{Phosphoglycerate mutase} & \underline{-6.84} & \underline{-4.93} & \underline{-3.80} & \underline{-2.78} & \underline{} & \underline{} & \underline{}\\
\hspace{1em}GRP2 & NAD(H)-linked methylglyoxal oxidoreductase & ex & -7.45 & -2.46 & -1.42 &  &  & \\
\hspace{1em}HSP21 & Small heat shock protein & ex & -4.97 & ex & -2.24 &  &  & \\
\hspace{1em}MET15 & O-acetylhomoserine O-acetylserine sulfhydrylase & ex & -2.91 & -2.00 & -2.72 &  &  & \\
\hspace{1em}orf19.3053 & Protein of unknown function & ex & -4.71 & -2.35 & -3.02 &  &  & \\
\hspace{1em}orf19.590 & Putative thiamine biosynthesis enzyme & ex & ex & ex & ex &  &  & \\
\hspace{1em}PGI1 & Glucose-6-phosphate isomerase & -6.91 & -5.44 & -4.14 & -1.68 &  &  & \\
\hspace{1em}PST2 & Flavodoxin-like protein & ex & ex & ex & -2.07 &  &  & \\
\hspace{1em}RIB3 & 3,4-Dihydroxy-2-butanone 4-phosphate synthase & ex & -1.06 & -1.23 & -2.11 &  &  & \\
\addlinespace[0.3em]
\multicolumn{9}{l}{\textbf{Proteasome}}\\
\hspace{1em}orf19.2755 & \textit{S. cerevisiae} ortholog is Pre7, Beta 6 subunit of 20S proteasome<sup>a</sup> & ex & -2.34 & ex & ex &  &  & \\
\hspace{1em}orf19.4230 & 20S proteasome subunit (beta7)<sup>a</sup> & ex & ex & ex & ex &  &  & \\
\hspace{1em}PUP2 & Alpha5 subunit of the 20S proteasome & ex & -4.13 & -2.53 & ex &  &  & \\
\hspace{1em}SCL1 & Proteasome subunit YC7alpha<sup>a</sup> & ex & ex & ex & ex &  &  & \\
\addlinespace[0.3em]
\multicolumn{9}{l}{\textbf{Cytoplasmic stress granule}}\\
\hspace{1em}TIF11 & Translation initiation factor eIF1a<sup>a</sup> & ex & ex & ex & ex &  &  & \\
\hspace{1em}TMA19 & \textit{S. cerevisiae} ortholog is Tma19, ribosome associated protein<sup>a</sup> & ex & -3.05 & -2.74 & ex &  &  & \\
\addlinespace[0.3em]
\multicolumn{9}{l}{\textbf{Peroxisome}}\\
\hspace{1em}orf19.1433 & \textit{S. cerevisiae} ortholog is Lpx1, peroxisomal lipase<sup>a</sup> & ex & ex & ex & ex &  &  & \\
\addlinespace[0.3em]
\multicolumn{9}{l}{\textbf{Actin cortical patch}}\\
\hspace{1em}\underline{ABP1} & \underline{Actin-binding protein} & \underline{ex} & \underline{-1.47} & \underline{ex} & \underline{ex} & \underline{} & \underline{} & \underline{}\\
\addlinespace[0.3em]
\multicolumn{9}{l}{\textbf{Unknown}}\\
\hspace{1em}orf19.2125 & Protein of unknown function<sup>c</sup> & ex & ex & ex & ex & 1 &  & \\
\hspace{1em}orf19.2737 & Carbohydrate kinase domain-containing protein<sup>c</sup> & ex & ex & ex & -1.48 &  &  & \\
\hspace{1em}orf19.5620 & Stationary phase enriched protein<sup>c</sup> & ex & ex & ex & -3.43 &  &  & \\
\hspace{1em}OYE32 & NADPH oxidoreductase family protein<sup>c</sup> & ex & ex & ex & -2.90 &  &  & \\
\bottomrule
\insertTableNotes
\end{longtable}
\end{ThreePartTable}
\endgroup{}
